\documentclass[letterpaper,10pt]{article}
\usepackage{fullpage}
\usepackage{url}
\newcommand{\Term}{Spring 2016}
\newcommand{\Office}{Monday 3-4:30}
\newcommand{\ExamOne}{Final Exam:  June 7th, 8:00-9:50}
\newcommand{\Section}{001}
%opening
\title{
Introductory Microeconomics\\
EC 201-\Section \\
\Term\\
}
\author{James Woods}
\date{}

\begin{document}

\maketitle

\section{Introduction}

Welcome to Microeconomics! Microeconomics is the study of individual
choice in the face of scarcity. In this course we will examine the
consequences of individuals and firms making choice about consumption
and production. We will also explore competitive behaviors and public
choice.

\section{Prerequisites}

There are no prerequisites for this course. However, all students are
assumed to have a working knowledge of high school algebra. I am much
less concerned with computation than I am with visualization, i.e.,
picturing the relationship between variables graphically.


\section{Contact Information}
If each of you sent me an email, text, or message, once a week, I would be unable to respond to all your questions.  Rather than emailing questions, I encourage you to post your questions on Piazza. You can find our class page at: \url{http://piazza.com/pdx/spring2016/ec201/home} .  There is also a link in D2L. If you have any problems or feedback for the developers, email team@piazza.com.

Please get signed up as soon as possible. I have set up a
folder for each of the class modules. If you want to discuss the class material, need a clarification or have a concern about a quiz question, look to see if someone has asked the same question and if it has been answered. 
If you have a question about course mechanics, interpretation of the
syllabus, or where something is located, there is a folder for
that called ``logistics''. I will check piazza on a regular basis during regular business hours, and answer questions. I also expect you and your peers to help to each other and will indicate the good answers and make clarifications.  

If you send me an email with a question that should be asked in a piazza, I
will ask you to post it there rather than answer in email.

The TA assigned to the class is TBA. 
 His contact information is:
 \begin{itemize}
 \item Office Hours: TBA, CH 230

 \item Email/IM: TBA

 \end{itemize}


My office is in CH 241-O.  The best ways of contacting me, in
decreasing order of effectiveness:
\begin{itemize}

\item IM/hangout: woodsj@pdx.edu

\item Phone/text: (503) 465-4883
\item Email:woodsj@pdx.edu (Answered only on M W and F)
\item D2L email: Do not use.  I will not respond.
\end{itemize}

You are encouraged to use the IM function of your PSU email account rather than email. This will allow me to get back to you more quickly and more conversationally. Please be aware that I will likely not reply till the next day if you IM me after 6pm or on the weekend.  

My office hours will be held \Office ~through the last week of class, not including finals week. There is no need to make an appointment for these hours -- just come.  There will be a link to a google hangout posted in D2L so you can attend office hours via webcam or in-person.

If you can't attend regular office hours, please check my calendar link \url{http://woodsj.youcanbook.me/}. I will make a limited number of 15 minute slots available each week. Appointments after 5pm will be via google hangout only. \emph{If you make an appointment and fail to show up without first canceling, I will penalize you one homework grade.}  


You will get a quicker response from both TBA and me if you include `EC201' in the subject line.


If class is canceled, I will post an announcement on Piazza.

\subsection{Attendance}

Please prepare by completing the required readings and reviewing your notes from the
previous lecture. My teaching style is very Socratic and requires a
lot of interaction with students to be effective. Not attending class
or coming unprepared reduces the quality of the class for all
students.

If you choose not to attend a class you are still responsible for
acquiring notes, handouts, and any announced schedule changes from
other students. You will also be ineligible for the debriefs described in section \ref{Grades}

\subsection{Books and Other Sources}
I have chosen to keep textbook costs to an absolute minimum.  The text, Rittenberg \& Tregarthen, ``Principles of Microeconomics, v. 2.0'' is available at \url{https://archive.org/details/fwk-archive-20121229-5230}.  

This text will be supplemented with material posted on D2L, often in the form of short videos and handouts.

\section{Exams and Assignments}\label{Grades}

Your grade in the class will be based on your performance on the final
exam, homework and your homework debriefs. The contribution of each of these to your final grade is shown below:

\begin{tabular}{ll}
  Final Exam&30\%\\
  Homework Debrief&20\%\\
  Homework&50\%\\
\end{tabular} 

Written homework will be due \emph{at least} weekly.  The homework will be made available in D2L as a dropbox with a specific due date.  
Late homework will not be accepted.  Your answers should be uploaded as a pdf to minimize problems with formatting and unusual file formats. Homework with technical problems, e.g., can't not be opened or missing attachment, will be marked as missing and receive zero credit.

Each homework question will be evaluated on a zero to two scale, with one being the most common score and two indicating an exceptional answer.  Just to be clear, it is possible to turn in a question and receive a score of zero.

The homework debrief is a group exercise conducted in class. Examples of the responses to the last homework will be handed out in class.  Each group will discuss:

\begin{itemize}
\item Which of the responses represents the best answer to the homework question.
\item How the other responses could be improved.
\item What was the likely misunderstanding in the other responses.
\end{itemize}

The group will recieve a grade on a zero to two scale similar to the one used on the homework.

The final exam will be comprehensive, covering all the material from the first day till the last day of lecture. The exam will consist of a subset of the homework questions that have already been assigned with some minor changes so rote memorization is not that effective.  The standards for acceptable and exceptional answers are significantly higher on the final exam.  These questions will be graded on a zero to five scale.  If you can not be present for the final exam, please consider enrolling in another class.


Grades will be based on your class rank. At the end of the term I will
create a weighted average score and rank the students. The dividing
lines between letter grades will be drawn such that no student is near
a dividing line. In this way no student will ever be, ``just one point
from an A.'' You are in a very real competition for grades in this
class. 

\subsubsection{Other Rules}
\begin{itemize}

\item No early or late final exams.

\item Begging for grades will result in an immediate lowering of your
  course grade by a full letter grade.

\item Go to office hours at the first sign of trouble -- not as a last
  resort.

\item In this classroom, we support and value diversity.  To do so requires that we:
\begin{itemize}
   \item Respect the dignity and essential worth of all individuals
   \item Promote a culture of respect toward all individuals
    \item Respect the privacy, property, and freedom of others
    \item Reject bigotry, discrimination, violence, or intimidation of any kind
    \item Practice personal and academic integrity and expect it from others
   \item Promote the diversity of opinions, ideas, and backgrounds, which is
    the lifeblood of a university
\end{itemize}

 For additional information, please see the Office of Affirmative Action \& Equal Opportunity at \url{http://www.pdx.edu/diversity/affirmative-action}.


\item Accommodations are collaborative efforts between students, faculty, and the Disability Resource Center.  If you have a documented disability and require accommodation, you must arrange to meet with the course instructor prior to or within the first week of the term.  The documentation of your disability must come in writing from the Disability Resource Center (Faculty letter).  Students who believe they are eligible for accommodations but who have not yet obtained approval through the DRC should contact the DRC immediately.  Reasonable and appropriate accommodations will be provided for students with documented disabilities.  For more information on the Disability Resource Center, please see \url{http://www.drc.pdx.edu/}. 


\item Academic honesty is expected and required of students enrolled
  in this course.  Suspected academic dishonesty in this course will
  be handled according to the procedures set out in the Student Code
  of Conduct.

\item I am sympathetic to family emergencies but you must inform me as
  soon as possible. If the notice is verbal, please email me with your
  understanding of our agreement. All agreements have to be in
  writing.
  
\item I expect homework to be completed by you without consulting your classmates or other human resources.  I take this seriously.  

\item Don't let your children be a barrier to attending class. Bringing your children to class occasionally is tolerated and encouraged.

% \item If you have not done so already, please complete the Safe Campus Module in d2l. The module should take approximately 30 to 40 minutes to complete and contains important information and resources. If you or someone you know has been harassed or assaulted, you can find the appropriate resources on PSU’s Enrollment Management \& Student Affairs: Sexual Prevention \& Response website at \url{http://www.pdx.edu/sexual-assault/}. PSU's Student Code of Conduct makes it clear that violence and harassment based on sex and gender are strictly prohibited and offenses are subject to the full realm of sanctions, up to and including suspension and expulsion. 

\end{itemize}

 
\subsection{Expected Outline}
A detailed and updated outline and due dates of reading and homework assignments will be available on D2L.  That schedule may be changed without notice.

\begin{itemize}
\item How to study economics: Notes from a successful student.
\item Mixing commands, tradition and doing things because you think it
  is a good idea, i.e., the real world. 
\item Why we trade?  It's easier than doing everything yourself.
\item Ideal markets: What they do, how rare they really are, and why
  it shouldn't bother you.
\item Messing with ideal markets is bad and messing with non-ideal
  markets could be good.
\item Measuring happiness with money!?
%\item Midterm
\item Where did that supply and demand stuff come from?  Costs.
\item Reasons to hate monopolies.
\item Reasons to hate things that are kind of like monopolies.
\item Three economic `Games' you see every day and what they have to
  do with pricing decisions.
\item What happens when someone else pays the bill.
\item What happens when we have to share.
\item Behavioral Economics
\end{itemize}

\medskip

\centering

\emph{\ExamOne}


\end{document}
